In conclusion, by uniting symbolic representations with entropic dynamics in a manifold-based framework, we reveal new principles underlying cognitive organization. This integrative perspective—spanning neuroscience, artificial intelligence and psychiatry—carries broad translational implications: from informing novel therapeutic strategies and interpretable AI algorithms to guiding evidence-based neuroeducation. Accordingly, we propose ``topological symbolic psychiatry'' as a new discipline grounded in this framework, conceptualizing mental disorders as distortions in a symbolic manifold and opening avenues for novel interventions. Such a perspective provides a metacognitive anchor by linking subjective meaning with quantitative topology, thereby bridging the gap between lived experience and scientific theory. By treating the mind as a navigable symbolic landscape shaped by entropic forces, we move towards a more unified science of cognition---and ultimately a transformative paradigm for understanding and improving human well-being.