\section*{Symbolic Cognitive Topology}

We introduce a symbolic manifold $\mathbb{S}$ in which cognitive activity evolves as a dynamic trajectory $\gamma(t) = (\alpha(t), \kappa(t), E_r(t))$. This formal structure is governed by three symbolic-cognitive variables:
\begin{itemize}
    \item \textbf{Anchoring coefficient} ($\alpha$): captures semantic and contextual coherence—higher values indicate well-integrated, focused thought; low values correspond to disorganization or fragmentation.
    \item \textbf{Symbolic curvature} ($\kappa$): quantifies the nonlinearity or divergence of associative links—analogous to cognitive flexibility, originality, or conceptual jumps.
    \item \textbf{Recursive entropy} ($E_r$): reflects the self-referential depth or symbolic unpredictability—higher $E_r$ signals increased symbolic recursion or chaotic semantic chaining.
\end{itemize}

The cognitive state evolves as a curve in $\mathbb{S}$ modulated by an entropic operator $\mathcal{E}$, defined qualitatively as:

\[
\frac{d\gamma}{dt} = \mathcal{E}(\gamma(t)) = 
\begin{pmatrix}
    -\eta_\alpha \cdot \alpha(t) + \phi_\alpha(\kappa,E_r) \\
    \eta_\kappa \cdot (1 - \alpha(t)) + \phi_\kappa(E_r) \\
    \eta_r \cdot \kappa(t) + \phi_r(\alpha)
\end{pmatrix}
\]

Here, $\eta_\alpha$, $\eta_\kappa$, and $\eta_r$ are modulation coefficients, and $\phi$ terms represent nonlinear interactions between dimensions. This captures how reductions in anchoring can trigger associative divergence or recursive overload.

Drawing from low-dimensional neural manifolds \cite{Langdon2023,Sizemore2019,Rouse2023} and symbolic attractor theory \cite{Helmich2021,Rolls2021}, we define regions in $\mathbb{S}$ corresponding to recurrent symbolic profiles, termed \textit{topotypes}. These are not static types but dynamic attractor basins, characterized by:

\begin{itemize}
    \item \textbf{Neurotypical} (high $\alpha$, low $E_r$, moderate $\kappa$)
    \item \textbf{Gifted cognition} (moderate $\alpha$, high $\kappa$, high $E_r$)
    \item \textbf{Twice-exceptional} (oscillatory $\alpha$, episodic surges in $E_r$)
    \item \textbf{Collapse} (declining $\alpha$, divergent $\kappa$, escalating $E_r$)
\end{itemize}

These profiles represent metastable regimes navigable over time, rather than fixed diagnostic labels. Time is discretized, and update rules allow for deterministic or stochastic simulation of $\gamma(t)$ trajectories.

This formulation integrates symbolic instability into the entropic brain hypothesis \cite{CarhartHarris2014} and REBUS model \cite{CarhartHarris2019}, embedding symbolic breakdowns into dynamic manifold transitions.

The model supports generative simulation and classification of symbolic-cognitive profiles, offering a formal framework for investigating neurodiversity, breakdowns, and symbolic restoration.