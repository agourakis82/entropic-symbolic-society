


\section*{A Symbolic Cognitive Topology}

The proposed model conceptualizes mental activity as a trajectory within a symbolic manifold—a topological space structured by three interdependent parameters:

\begin{itemize}
  \item $\alpha(t)$: the \textbf{anchoring coefficient}, representing the degree to which a cognitive agent maintains semantic coherence, context stability, and internal referential anchoring over time;
  \item $\kappa(t)$: the \textbf{symbolic curvature}, quantifying the complexity, density, and nonlinearity of symbolic associations within the agent's representational field;
  \item $E_r(t)$: the \textbf{recursive entropy}, reflecting the rate and unpredictability of symbolic feedback loops, re-entries, and self-referential activation patterns.
\end{itemize}

Together, these parameters define a point $\gamma(t) = (\alpha(t), \kappa(t), E_r(t))$ in the symbolic phase space $\mathbb{S}$. Cognitive evolution is treated as a continuous function $\gamma: \mathbb{R}^+ \rightarrow \mathbb{S}$, constrained by entropic and symbolic forces.

This symbolic topology is governed by an entropic operator $\mathcal{E}$, which modulates the agent's trajectory in response to internal dynamics and external perturbations. Formally, we define:

\[
\frac{d\gamma}{dt} = \mathcal{E}(\gamma(t)) = \left( -\frac{\partial \alpha}{\partial t}, \frac{\partial \kappa}{\partial t}, \frac{\partial E_r}{\partial t} \right)
\]

Under typical conditions, $\alpha$ exhibits damping behavior, $\kappa$ fluctuates nonlinearly with semantic novelty, and $E_r$ increases with symbolic recursion depth. However, specific regions of the manifold act as attractors or bifurcation zones, producing unique cognitive profiles.

We identify \emph{topotypes}—distinct symbolic signatures of cognitive organization—as stable or metastable regions in $\mathbb{S}$:

\begin{itemize}
  \item Neurotypical cognition: high $\alpha$, moderate $\kappa$, low $E_r$
  \item Gifted cognitive singularity: moderate $\alpha$, high $\kappa$, high $E_r$
  \item 2e (twice-exceptional) oscillatory singularity: low-to-moderate $\alpha$, fluctuating $\kappa$, episodically high $E_r$
  \item Entropic collapse states (e.g., psychosis): unstable $\alpha$, divergent $\kappa$, escalating $E_r$
\end{itemize}

These topotypes are not static categories, but dynamic configurations: a single mind may shift across regions of $\mathbb{S}$ depending on internal state and environmental input. This dynamical formulation allows for the modeling of state transitions, diagnostic thresholds, and singularities of high creativity or breakdown.

To enable simulation and reparameterization, we discretize time and define update rules for each parameter based on symbolic energy flux and anchoring decay. The model supports both deterministic and stochastic implementations, depending on the desired granularity of simulation.

In sum, this section formalizes a symbolic phase space where cognition unfolds as an entropically modulated trajectory. This topological model enables classification, simulation, and potential clinical application for understanding high-variance cognitive phenomena.