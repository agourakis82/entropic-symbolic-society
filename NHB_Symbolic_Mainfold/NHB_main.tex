\documentclass[12pt]{article}
\usepackage[a4paper, margin=2.5cm]{geometry}
\usepackage{amsmath, amssymb, graphicx, cite}
\usepackage{hyperref}
\usepackage{authblk}
\usepackage{titlesec}
\usepackage{float}

% Metadata
\title{Symbolic Manifolds and Entropic Dynamics:\\ A Cognitive Topology of Mental States}
\author[1,2]{Demetrios Chiuratto Agourakis}
\affil[1]{Faculdade São Leopoldo Mandic, Campinas, SP, Brazil}
\affil[2]{Postgraduate Program in Biomaterials and Regenerative Medicine, PUC-SP, Brazil}
\date{}

\begin{document}
\maketitle

\% Abstract (updated after version 1.2 Zenodo DOI: 10.5281/zenodo.16682785)
\begin{abstract}
\vspace{0.3cm}
We introduce a novel framework for modeling mental states through symbolic manifolds, coupling entropy-driven dynamics with topological structures across multiscale cognitive regimes. By simulating trajectories $\gamma(t) = (\alpha, \kappa, E_r)$, we demonstrate bifurcation, collapse-recovery phenomena, and symbolic regime transitions across neurotypical, gifted, twice-exceptional, and collapse-prone cognitive configurations. The model is grounded in high-dimensional symbolic cognition, integrating empirical inspiration from psychiatric models and theoretical anchors in entropy and dynamic systems theory. Our results position symbolic manifolds as a viable, interpretable topology for bridging computational psychiatry, neurodiversity, and symbolic cognition.
\end{abstract}

% Introduction
\section*{Introduction}


\section*{Introduction}

Human cognition unfolds across a multidimensional landscape shaped not only by neurons and neurotransmitters, but also by symbols, narratives, and recursive self-reference. Despite increasing interest in the dynamics of consciousness, creativity, and neurodiversity, current cognitive models often remain confined to linear or categorical frameworks, unable to fully capture the recursive, context-dependent nature of mental states.

In recent years, research has begun to explore cognition as a dynamic system governed by principles of complexity and entropy \cite{carhart2014entropic, bassett2017network}. Yet, even these approaches often overlook the symbolic architecture of mind—the way in which individuals anchor meaning, generate representational curvature, and navigate between internal coherence and external adaptation.

Here we introduce a theoretical framework based on symbolic manifolds: a cognitive topology defined by three entropic-symbolic parameters—anchoring coefficient ($\alpha$), symbolic curvature ($\kappa$), and recursive entropy ($E_r$). Together, these variables describe the evolving state of a cognitive system as it traverses a symbolic phase space. The proposed model formalizes the intuition that high-level cognition is neither stable nor chaotic, but dynamically self-modulating across an entropic gradient.

To operationalize this framework, we derive a symbolic analogue of the Schrödinger equation, treating mental trajectories as $\gamma(t) = (\alpha(t), \kappa(t), E_r(t))$ evolving over time. Through simulation, we examine how different parameter configurations yield recognizable cognitive profiles—from neurotypical states to gifted and twice-exceptional (2e) singularities—each occupying distinct regions in the symbolic manifold.

Our goal is not merely to describe exceptional minds, but to propose a topology wherein such minds become computationally representable as structured dynamical attractors. This symbolic manifold offers a new lens for interpreting cognitive variability, bridging creativity, neurodivergence, and mental instability under a unified entropic formalism.

% Model and Simulations
\section*{Model and Simulations}

\subsection*{A Symbolic Cognitive Topology}
\section*{Symbolic Cognitive Topology}

% Revisão: reformulação das frases introdutórias para concisão e manutenção de estilo NHB
We propose a symbolic manifold $\mathbb{S}$ wherein cognitive activity evolves as a dynamic trajectory $\gamma(t) = (\alpha(t), \kappa(t), E_r(t))$. This state space is parameterized by three cognitive-symbolic variables:
\begin{itemize}
    \item \textbf{Anchoring coefficient} ($\alpha$): captures semantic and contextual coherence. High $\alpha$ denotes well-integrated, focused thought, whereas low $\alpha$ corresponds to disorganized or fragmented cognition.
    \item \textbf{Symbolic curvature} ($\kappa$): quantifies the nonlinearity or divergence of associative links. Higher $\kappa$ reflects greater cognitive flexibility or originality, marked by conceptual ``jumps'' in thought.
    \item \textbf{Recursive entropy} ($E_r$): measures the depth of self-referential processing or symbolic unpredictability. A higher $E_r$ indicates more pronounced symbolic recursion and chaotic semantic chaining.
\end{itemize}

% Revisão: descrição do operador entrópico refinada, mantendo notação consistente
We formalize the cognitive state’s evolution as a trajectory on $\mathbb{S}$ governed by an entropic operator $\mathcal{E}$, defined by:
\[
\frac{d\gamma}{dt} = \mathcal{E}(\gamma(t)) \;=\;
\begin{pmatrix}
    -\,\eta_\alpha\,\alpha(t) + \phi_\alpha(\kappa(t),\,E_r(t))\\
    \eta_\kappa\,\big(1 - \alpha(t)\big) + \phi_\kappa(E_r(t))\\
    \eta_r\,\kappa(t) + \phi_r(\alpha(t))
\end{pmatrix}\!,
\]
where $\eta_\alpha$, $\eta_\kappa$, and $\eta_r$ are coefficients modulating intrinsic linear dynamics, and each $\phi$ term encodes a specific nonlinear interaction between dimensions. For example, a reduction in anchoring ($\alpha$) increases the drive $\eta_\kappa[1-\alpha(t)]$ for associative divergence ($\kappa$) and can elevate recursive entropy $E_r$ via $\phi_r$, illustrating how loosening semantic coherence triggers divergent, self-amplifying thought streams.

% Revisão: expansão conceitual de 'topótipos' como atratores dinâmicos
Drawing inspiration from low-dimensional neural manifold models \cite{langdon2023, sizemore2019, rouse2023} and attractor-based cognitive dynamics \cite{helmich2021, rolls2021}, we delineate distinct regions of $\mathbb{S}$ corresponding to recurrent cognitive profiles, or \textit{topotypes}. These topotypes are not static categories but dynamic attractor basins in the symbolic manifold, each characterized by a prototypical $(\alpha, \kappa, E_r)$ regime:
\begin{itemize}
    \item \textbf{Neurotypical}: high $\alpha$, moderate $\kappa$, low $E_r$.
    \item \textbf{Gifted cognition}: moderate $\alpha$, high $\kappa$, high $E_r$.
    \item \textbf{Twice-exceptional (2e)}: oscillatory $\alpha$ (flutuando entre alto/baixo), intermediário $\kappa$, episódios de surto em $E_r$.
    \item \textbf{Collapse}: $\alpha$ em declínio contínuo, $\kappa$ em disparada (divergente) e $E_r$ escalando.
\end{itemize}

% Extensão: análise de estabilidade, bifurcação e sensibilidade a condições iniciais adicionada
Each topotype corresponds to a metastable regime that a cognitive trajectory can occupy or transition between over time, rather than a fixed label. Dynamically, these regimes act as attractors (point attractors or cyclic orbits) in the state space of $\mathbb{S}$. For instance, the Neurotypical profile may correspond to a stable fixed-point attractor (anchoring providing a strong restoring force), whereas the twice-exceptional profile might emerge from a limit cycle (periodic oscillations in $\alpha$ and $E_r$). Changes in parameters can induce qualitative regime shifts (bifurcations); for example, lowering the anchoring feedback $\eta_\alpha$ could destabilize the Neurotypical basin, precipitating a drift toward Collapse or an oscillatory 2e-like state. Such sensitivity to parameters and initial conditions is characteristic of nonlinear dynamical systems, echoing approaches in computational psychiatry that link altered attractor landscapes to shifts in mental state \cite{vandeleemput2014, gauld2023}. % Referências sugeridas: van de Leemput et al. 2014; Gauld & Depannemaecker 2023

% Detalhe: menção à discretização temporal e simulação estocástica
For simulations, we treat time in discrete steps (see Supplementary Information S1) and iteratively update the state as $\gamma_{t+1} = \gamma_t + \mathcal{E}(\gamma_t)$. This approach accommodates both deterministic and stochastic trajectories, where stochasticity is introduced by adding small noise terms in each dimension to mimic symbolic instability.

% Revisão: integração com hipóteses existentes e conclusão reformulada
Finally, our entropic manifold formulation integrates the notion of symbolic instability into the entropic brain hypothesis \cite{carhart2014entropic} and its REBUS model \cite{carhart2019rebus}, reframing cognitive breakdowns as transitions within a dynamic mental topology. Accordingly, the model supports generative simulations and classification of symbolic-cognitive profiles, providing a formal framework to investigate neurodiversity, cognitive breakdowns, and prospects for symbolic restoration.


\subsection*{Simulated Entropic Dynamics in Mental Manifolds}
\section*{3. Simulation Results}

To evaluate the expressive capacity of the symbolic manifold $\mathbb{S}$, we conducted discrete-time simulations of trajectories $\gamma(t) = (\alpha(t), \kappa(t), E_r(t))$ under idealized initial conditions representative of four cognitive-symbolic profiles: Neurotypical (NT), Gifted (G), Twice-exceptional (2e), and Collapse-prone (C). The system evolves via the entropic modulation operator $\mathcal{E}$ (see Section 2), incorporating deterministic nonlinear interactions and parametric noise.

\paragraph{Profile 1: Neurotypical (NT)} shows stable anchoring ($\alpha \approx 0.9$), low symbolic curvature ($\kappa \approx 0.1$), and minimal entropy ($E_r \approx 0.05$). Its trajectory remains confined to a narrow symbolic basin, suggesting low variance in symbolic state over time (Fig.~\ref{fig:gamma_profiles}).

\paragraph{Profile 2: Gifted (G)} exhibits moderate anchoring ($\alpha \approx 0.6$), elevated curvature ($\kappa \approx 0.75$), and high recursive entropy ($E_r \approx 0.8$). The trajectory demonstrates symbolic divergence without collapse, reflecting nonlinearity and symbolic flexibility (Fig.~\ref{fig:gamma_profiles}).

\paragraph{Profile 3: Twice-exceptional (2e)} oscillates between basins, with $\alpha(t)$ fluctuating between 0.3 and 0.85, and episodic surges in $E_r(t)$. This profile combines phases of symbolic coherence with entropic disruption, simulating cognitive dissonance or oscillatory stability (Fig.~\ref{fig:gamma_profiles}).

\paragraph{Profile 4: Collapse-prone (C)} starts at $\alpha_0 = 0.5$ and undergoes progressive degradation due to compounding $E_r$ and $\kappa$. By $t = 50$, $\alpha \rightarrow 0.1$, and $E_r > 0.9$, indicating a symbolic breakdown. This regime reflects unstable symbolic anchoring and associative overload.

\paragraph{Collapse–Recovery Scenario:} We simulated an external dampening event at $t = 70$ reducing $E_r$ and increasing $\alpha$ by intervention terms $\delta_\alpha$, $\delta_E$. The trajectory returns to a bounded symbolic region, illustrating restoration of coherence under entropic compression (Fig.~\ref{fig:collapse_recovery}).

\paragraph{Symbolic Topology:} A 3D trajectory $\gamma(t)$ over $(\alpha, \kappa, E_r)$ reveals organized transitions and clustering (Fig.~\ref{fig:trajectory_3D}), suggesting symbolic attractors and the potential for symbolic fingerprinting across individuals.

\paragraph{Comparative Positioning:} Table~\ref{tab:model_comparison} situates our framework relative to other cognitive theories. While FEP and Entropic Brain offer entropy-centric formulations, the symbolic manifold uniquely enables symbolic interpretability and neurodiversity modeling.

Together, these simulations support the hypothesis that symbolic cognitive profiles manifest as structured dynamic regimes in $\mathbb{S}$, offering a generative and interpretable approach to modeling variation in mental states.

% Discussion
\section*{Discussion and General Synthesis}
\section*{Discussion and General Synthesis}

This study introduced a symbolic manifold model that formalizes cognitive variation as trajectories $\gamma(t)$ evolving through a three-dimensional symbolic space: anchoring ($\alpha$), symbolic curvature ($\kappa$), and recursive entropy ($E_r$). By representing cognitive profiles as dynamic positions within this manifold, the model provides a unified framework for simulating transitions across states such as neurotypicality, gifted cognition, oscillatory dual profiles (2e), and symbolic collapse.

Simulation results demonstrate that each profile traces a structured and parameter-sensitive trajectory in $\mathbb{S}$. These symbolic signatures reflect stable or metastable configurations in symbolic space and suggest that neurodiversity can be modeled as differences in symbolic dynamics rather than as categorical distinctions. The model supports classification, visualization, and intervention hypotheses: targeted modulation of $\alpha$, $\kappa$, or $E_r$ may enable theoretical restoration of symbolic coherence, paralleling cognitive-behavioral scaffolding or entropic modulation therapies.

Importantly, this framework reconceptualizes neurodivergence not as disorder but as structured displacement within a symbolic topological landscape. Gifted and 2e trajectories display increased curvature and entropy while preserving anchoring intermittently, modeling creativity and cognitive oscillation. Collapse regimes, conversely, show runaway entropy and anchoring loss—suggesting that breakdowns may be framed not as categorical failures but as failed self-organization in symbolic space.

Several limitations merit attention. The symbolic variables proposed are currently theoretical and require future empirical grounding. Potential proxies include: coherence metrics for $\alpha$ (e.g., topic continuity in speech), network divergence measures for $\kappa$ (e.g., semantic distance in conceptual associations), and entropy rate in symbolic generation for $E_r$. Additionally, while simulations qualitatively reproduce plausible patterns, further mathematical formalization and parameter calibration are needed to support generalizability and reproducibility.

Future directions include the application of this model to longitudinal cognitive data (e.g., semantic dynamics, thought disorder profiles, neuroimaging of functional connectivity under symbolic tasks) to empirically test the model's explanatory and predictive power. Moreover, a formal symbolic-to-neural mapping—via, for example, neuro-symbolic transformers or latent embedding alignment—could bridge this manifold with real-time brain dynamics.

In sum, the symbolic manifold offers a generative, falsifiable, and interpretable structure to reconceptualize cognitive states through the lens of symbolic entropy. By embedding diversity, instability, and transformation within a unified topological grammar, this work contributes to the formal modeling of mind as a structured entropic system—opening new paths for interdisciplinary research across cognitive science, psychiatry, and neurocomputational theory.

% Methods
\section*{Methods}
% 05_methods.tex  — methods with stability analysis & glossary added
\section*{Methods}

% Existing modelling sections here ...

\subsection*{Formal Stability Analysis}
To characterise the system's attractor structure, we derived the equilibrium points $(\alpha^{\ast},\kappa^{\ast},E_r^{\ast})$ by solving Eqs.~1–3 for $\dot{\alpha}=\dot{\kappa}=\dot{E}_r=0$.
The Jacobian of the deterministic system evaluated at equilibrium is
\[
\mathbf{J}=
\begin{pmatrix}
-\eta_{\alpha}-K_{\kappa}\kappa^{\ast}-3\gamma_{\alpha}{\alpha^{\ast}}^{2} &
-K_{\kappa}\alpha^{\ast} &
\displaystyle\frac{K_E\theta_E}{(\theta_E+E_r^{\ast})^{2}}
\\
U &
a-\eta_{\kappa}-3b{\kappa^{\ast}}^{2} &
-V
\\
-X &
-Y &
-\eta_{r}
\end{pmatrix}.
\]
A fixed point is locally stable when $\Re\lambda_i<0$ for all eigenvalues $\lambda_i$ of $\mathbf{J}$.
Numerically, with default parameters (Extended Data Table 1) the neurotypical equilibria have $\lambda\approx\{-1.16\pm0.02i,-1.00\}$, confirming spiral stability, whereas collapse‑prone states possess one positive real eigenvalue ($\lambda_{1}\approx0.07$), acting as saddles.

\subsection*{Bifurcation with hysteresis}
Varying the basal entropic drive $W$ uncovers a saddle‑node bifurcation with hysteresis (Fig.~\ref{fig:bifurcation-hyst}). Increasing $W$ past $W_{\mathrm{up}}\approx1.1$ collapses $\alpha$, whereas recovery occurs only when $W$ drops below $W_{\mathrm{down}}\approx0.7$.

% Glossary box
\begin{table}[t]
\caption{\textbf{Glossary of key terms.}}
\label{tab:glossary}
\renewcommand{\arraystretch}{1.15}
\begin{tabular}{p{3cm}p{12cm}}
\textbf{Symbol / term} & \textbf{Definition} \\ \hline
$\alpha$ & Anchoring coefficient; semantic coherence of thought. \\
$\kappa$ & Symbolic curvature; divergence of associative thought. \\
$E_r$ & Recursive entropy; unpredictability / novelty of cognitive flow. \\
Symbolic manifold & 3‑D state‑space $(\alpha,\kappa,E_r)$ for cognition. \\
Topotype & Metastable attractor regime in the manifold. \\ \hline
\end{tabular}
\end{table}


% References
\bibliographystyle{unsrt}
\bibliography{references}

\end{document}
