\documentclass[12pt]{article}
\usepackage[a4paper, margin=2.5cm]{geometry}
\usepackage{amsmath, amssymb, graphicx, cite}
\usepackage{hyperref}
\usepackage{authblk}
\usepackage{titlesec}
\usepackage{float}

% Metadata
\title{Symbolic Manifolds and Entropic Dynamics:\\ A Cognitive Topology of Mental States}
\author[1,2]{Demetrios Chiuratto Agourakis}
\affil[1]{Faculdade São Leopoldo Mandic, Campinas, SP, Brazil}
\affil[2]{Postgraduate Program in Biomaterials and Regenerative Medicine, PUC-SP, Brazil}
\date{}

\begin{document}
\maketitle

\% Abstract (updated after version 1.2 Zenodo DOI: 10.5281/zenodo.16682785)
\begin{abstract}
\vspace{0.3cm}
We introduce a novel framework for modeling mental states through symbolic manifolds, coupling entropy-driven dynamics with topological structures across multiscale cognitive regimes. By simulating trajectories $\gamma(t) = (\alpha, \kappa, E_r)$, we demonstrate bifurcation, collapse-recovery phenomena, and symbolic regime transitions across neurotypical, gifted, twice-exceptional, and collapse-prone cognitive configurations. The model is grounded in high-dimensional symbolic cognition, integrating empirical inspiration from psychiatric models and theoretical anchors in entropy and dynamic systems theory. Our results position symbolic manifolds as a viable, interpretable topology for bridging computational psychiatry, neurodiversity, and symbolic cognition.
\end{abstract}

% Introduction
\section*{Introduction}
\section*{Introduction}

Understanding the dynamic nature of cognition remains a central challenge in cognitive science. Traditional models often rely on static categories or linear representations, which fail to capture the fluid, recursive, and symbolically mediated nature of mental processes—particularly in cases of neurodivergence, exceptional cognitive ability, or disorganization.

Recent research has highlighted the role of entropy and complexity in shaping brain activity and cognitive function \cite{carhart2014entropic, bassett2017network}, suggesting that cognition emerges from nonlinear, metastable dynamics. Yet, these frameworks frequently overlook the influence of symbolic structure, contextual grounding, and self-referential feedback in shaping cognitive states.

Here, we propose a novel theoretical framework based on a symbolic manifold—a topological space parameterized by three cognitive-symbolic variables: anchoring coefficient ($\alpha$), symbolic curvature ($\kappa$), and recursive entropy ($E_r$). These variables jointly describe the evolving state of a cognitive agent as a trajectory $\gamma(t)$ in symbolic space, governed by interactions between symbolic coherence, conceptual divergence, and entropic modulation.

Rather than viewing mental states as discrete or pathologized categories, we conceptualize them as structured positions and flows within this manifold. Cognitive profiles—such as neurotypical, gifted, and twice-exceptional (2e)—emerge as attractor regimes, each exhibiting distinct dynamic signatures.

This model enables the simulation of symbolic trajectories and the mapping of cognitive diversity within a unified, generative structure. By integrating symbolic topology with entropic principles, we aim to provide a rigorous and extensible basis for representing cognitive variability, with potential applications in neuropsychology, psychiatry, and computational modeling.

% Model and Simulations
\section*{Model and Simulations}

\subsection*{A Symbolic Cognitive Topology}



\section*{A Symbolic Cognitive Topology}

The proposed model conceptualizes mental activity as a trajectory within a symbolic manifold—a topological space structured by three interdependent parameters:

\begin{itemize}
  \item $\alpha(t)$: the \textbf{anchoring coefficient}, representing the degree to which a cognitive agent maintains semantic coherence, context stability, and internal referential anchoring over time;
  \item $\kappa(t)$: the \textbf{symbolic curvature}, quantifying the complexity, density, and nonlinearity of symbolic associations within the agent's representational field;
  \item $E_r(t)$: the \textbf{recursive entropy}, reflecting the rate and unpredictability of symbolic feedback loops, re-entries, and self-referential activation patterns.
\end{itemize}

Together, these parameters define a point $\gamma(t) = (\alpha(t), \kappa(t), E_r(t))$ in the symbolic phase space $\mathbb{S}$. Cognitive evolution is treated as a continuous function $\gamma: \mathbb{R}^+ \rightarrow \mathbb{S}$, constrained by entropic and symbolic forces.

This symbolic topology is governed by an entropic operator $\mathcal{E}$, which modulates the agent's trajectory in response to internal dynamics and external perturbations. Formally, we define:

\[
\frac{d\gamma}{dt} = \mathcal{E}(\gamma(t)) = \left( -\frac{\partial \alpha}{\partial t}, \frac{\partial \kappa}{\partial t}, \frac{\partial E_r}{\partial t} \right)
\]

Under typical conditions, $\alpha$ exhibits damping behavior, $\kappa$ fluctuates nonlinearly with semantic novelty, and $E_r$ increases with symbolic recursion depth. However, specific regions of the manifold act as attractors or bifurcation zones, producing unique cognitive profiles.

We identify \emph{topotypes}—distinct symbolic signatures of cognitive organization—as stable or metastable regions in $\mathbb{S}$:

\begin{itemize}
  \item Neurotypical cognition: high $\alpha$, moderate $\kappa$, low $E_r$
  \item Gifted cognitive singularity: moderate $\alpha$, high $\kappa$, high $E_r$
  \item 2e (twice-exceptional) oscillatory singularity: low-to-moderate $\alpha$, fluctuating $\kappa$, episodically high $E_r$
  \item Entropic collapse states (e.g., psychosis): unstable $\alpha$, divergent $\kappa$, escalating $E_r$
\end{itemize}

These topotypes are not static categories, but dynamic configurations: a single mind may shift across regions of $\mathbb{S}$ depending on internal state and environmental input. This dynamical formulation allows for the modeling of state transitions, diagnostic thresholds, and singularities of high creativity or breakdown.

To enable simulation and reparameterization, we discretize time and define update rules for each parameter based on symbolic energy flux and anchoring decay. The model supports both deterministic and stochastic implementations, depending on the desired granularity of simulation.

In sum, this section formalizes a symbolic phase space where cognition unfolds as an entropically modulated trajectory. This topological model enables classification, simulation, and potential clinical application for understanding high-variance cognitive phenomena.

\subsection*{Simulated Entropic Dynamics in Mental Manifolds}
\section*{3. Simulation Results}

To evaluate the expressive capacity of the symbolic manifold $\mathbb{S}$, we conducted discrete-time simulations of trajectories $\gamma(t) = (\alpha(t), \kappa(t), E_r(t))$ under idealized initial conditions representative of four cognitive-symbolic profiles: Neurotypical (NT), Gifted (G), Twice-exceptional (2e), and Collapse-prone (C). The system evolves via the entropic modulation operator $\mathcal{E}$ (see Section 2), incorporating deterministic nonlinear interactions and parametric noise.

\paragraph{Profile 1: Neurotypical (NT)} shows stable anchoring ($\alpha \approx 0.9$), low symbolic curvature ($\kappa \approx 0.1$), and minimal entropy ($E_r \approx 0.05$). Its trajectory remains confined to a narrow symbolic basin, suggesting low variance in symbolic state over time (Fig.~\ref{fig:gamma_profiles}).

\paragraph{Profile 2: Gifted (G)} exhibits moderate anchoring ($\alpha \approx 0.6$), elevated curvature ($\kappa \approx 0.75$), and high recursive entropy ($E_r \approx 0.8$). The trajectory demonstrates symbolic divergence without collapse, reflecting nonlinearity and symbolic flexibility (Fig.~\ref{fig:gamma_profiles}).

\paragraph{Profile 3: Twice-exceptional (2e)} oscillates between basins, with $\alpha(t)$ fluctuating between 0.3 and 0.85, and episodic surges in $E_r(t)$. This profile combines phases of symbolic coherence with entropic disruption, simulating cognitive dissonance or oscillatory stability (Fig.~\ref{fig:gamma_profiles}).

\paragraph{Profile 4: Collapse-prone (C)} starts at $\alpha_0 = 0.5$ and undergoes progressive degradation due to compounding $E_r$ and $\kappa$. By $t = 50$, $\alpha \rightarrow 0.1$, and $E_r > 0.9$, indicating a symbolic breakdown. This regime reflects unstable symbolic anchoring and associative overload.

\paragraph{Collapse–Recovery Scenario:} We simulated an external dampening event at $t = 70$ reducing $E_r$ and increasing $\alpha$ by intervention terms $\delta_\alpha$, $\delta_E$. The trajectory returns to a bounded symbolic region, illustrating restoration of coherence under entropic compression (Fig.~\ref{fig:collapse_recovery}).

\paragraph{Symbolic Topology:} A 3D trajectory $\gamma(t)$ over $(\alpha, \kappa, E_r)$ reveals organized transitions and clustering (Fig.~\ref{fig:trajectory_3D}), suggesting symbolic attractors and the potential for symbolic fingerprinting across individuals.

\paragraph{Comparative Positioning:} Table~\ref{tab:model_comparison} situates our framework relative to other cognitive theories. While FEP and Entropic Brain offer entropy-centric formulations, the symbolic manifold uniquely enables symbolic interpretability and neurodiversity modeling.

Together, these simulations support the hypothesis that symbolic cognitive profiles manifest as structured dynamic regimes in $\mathbb{S}$, offering a generative and interpretable approach to modeling variation in mental states.

% Discussion
\section*{Discussion and General Synthesis}
\section*{Discussion and General Synthesis}

This study introduced a symbolic manifold model that formalizes cognitive variation as trajectories $\gamma(t)$ evolving through a three-dimensional symbolic space: anchoring ($\alpha$), symbolic curvature ($\kappa$), and recursive entropy ($E_r$). By representing cognitive profiles as dynamic positions within this manifold, the model provides a unified framework for simulating transitions across states such as neurotypicality, gifted cognition, oscillatory dual profiles (2e), and symbolic collapse.

Simulation results demonstrate that each profile traces a structured and parameter-sensitive trajectory in $\mathbb{S}$. These symbolic signatures reflect stable or metastable configurations in symbolic space and suggest that neurodiversity can be modeled as differences in symbolic dynamics rather than as categorical distinctions. The model supports classification, visualization, and intervention hypotheses: targeted modulation of $\alpha$, $\kappa$, or $E_r$ may enable theoretical restoration of symbolic coherence, paralleling cognitive-behavioral scaffolding or entropic modulation therapies.

Importantly, this framework reconceptualizes neurodivergence not as disorder but as structured displacement within a symbolic topological landscape. Gifted and 2e trajectories display increased curvature and entropy while preserving anchoring intermittently, modeling creativity and cognitive oscillation. Collapse regimes, conversely, show runaway entropy and anchoring loss—suggesting that breakdowns may be framed not as categorical failures but as failed self-organization in symbolic space.

Several limitations merit attention. The symbolic variables proposed are currently theoretical and require future empirical grounding. Potential proxies include: coherence metrics for $\alpha$ (e.g., topic continuity in speech), network divergence measures for $\kappa$ (e.g., semantic distance in conceptual associations), and entropy rate in symbolic generation for $E_r$. Additionally, while simulations qualitatively reproduce plausible patterns, further mathematical formalization and parameter calibration are needed to support generalizability and reproducibility.

Future directions include the application of this model to longitudinal cognitive data (e.g., semantic dynamics, thought disorder profiles, neuroimaging of functional connectivity under symbolic tasks) to empirically test the model's explanatory and predictive power. Moreover, a formal symbolic-to-neural mapping—via, for example, neuro-symbolic transformers or latent embedding alignment—could bridge this manifold with real-time brain dynamics.

In sum, the symbolic manifold offers a generative, falsifiable, and interpretable structure to reconceptualize cognitive states through the lens of symbolic entropy. By embedding diversity, instability, and transformation within a unified topological grammar, this work contributes to the formal modeling of mind as a structured entropic system—opening new paths for interdisciplinary research across cognitive science, psychiatry, and neurocomputational theory.

% Methods
\section*{Methods}
% =========================================================
%  05_methods.tex  —  Methods section (Nature Human Behaviour)
% =========================================================
\section*{Methods}

\subsection*{Model Equations}

We model cognitive dynamics as trajectories $\gamma(t)=\bigl(\alpha(t),\kappa(t),E_{r}(t)\bigr)$ on a three‑dimensional symbolic manifold $\mathbb{S}$.  The governing ordinary differential equations are

\begin{align}
\dot{\alpha} &= \underbrace{K_E\,\frac{E_r}{E_r + \theta_E}\;-\;K_{\kappa}\,\kappa\,\alpha\;-\;\gamma_\alpha\,\alpha^{3}}_{\phi_\alpha(\alpha,\kappa,E_r)}
               \;-\;\eta_\alpha\,\alpha,\\[4pt]
\dot{\kappa} &= \underbrace{a\,\kappa\;-\;b\,\kappa^{3}\;+\;U\,\alpha\;-\;V\,E_r}_{\phi_\kappa(\alpha,\kappa,E_r)}
               \;-\;\eta_\kappa\,\kappa,\\[4pt]
\dot{E}_{r} &= \underbrace{W\;-\;X\,\alpha\;-\;Y\,\kappa}_{\phi_r(\alpha,\kappa,E_r)}
               \;-\;\eta_r\,E_r + \xi(t),
\end{align}

where the $\eta_i$ are linear dissipation constants, $K_E$–$Y$ are non‑linear gain parameters, and $\xi(t)$ is Gaussian white noise with zero mean and variance $\sigma_\xi^{2}$.

\subsection*{Simulation Procedure}

We solved the stochastic system using the Euler–Maruyama integrator ($\Delta t=10^{-2}$, $T=200$ a.u., seed = 42).  
All code (Python 3) is provided in the repository under \texttt{code/simulate\_collapse.py}; the core loop is reproduced in Supplementary Algorithm 1 for transparency.  
Figures \ref{fig:collapse} and \ref{fig:bifurcation} were generated with Matplotlib; each PNG is $1200\times800$ px (300 dpi).

\subsection*{Stability Analysis}

Nullclines were obtained analytically; Jacobian eigenvalues were computed numerically at each fixed point.  
A drive‑entropy bifurcation was mapped by slowly ramping $W$ (\SIrange{0.2}{1.2}{\bits\per\second}) and recording asymptotic states (Fig.~\ref{fig:bifurcation}).  
Full eigen‑spectra and Lyapunov exponents are reported in Supplementary Table S1.

\subsection*{Empirical Mapping}

Table \ref{tab:empirical} summarises putative empirical proxies:

\begin{table}[ht]
\centering
\caption{Proposed mappings between model variables and measurable biomarkers.}
\label{tab:empirical}
\begin{tabular}{@{}lll@{}}
\toprule
Variable & Primary proxy & Supporting refs \\
\midrule
$\alpha$ (symbolic anchoring) & Semantic coherence of free speech; DMN–hippocampus co‑activation & \cite{Bedi2015speech,Sporns2013network} \\
$\kappa$ (network coupling)   & Global efficiency / DMN centrality; PCC $\alpha$‑band power          & \cite{Bassett2017network,CarhartHarris2014} \\
$E_r$ (recursive entropy)     & Lempel–Ziv complexity of EEG/MEG; PCI\textsubscript{TMS–EEG}        & \cite{Schartner2017evidence,CarhartHarris2014} \\
\bottomrule
\end{tabular}
\end{table}

\subsection*{Parameter Values}

A complete list of parameter symbols, default values and empirical rationale is provided in Extended Data Table 1.  Key defaults used in all main‑text simulations are:

\begin{center}
\begin{tabular}{@{}lcc@{}}
\toprule
Parameter (unit) & Default & Role \\
\midrule
$K_E$ (—)       & 2.0 & Entropic drive on $\alpha$ \\
$K_{\kappa}$ (—)& 0.5 & Inhibition of $\alpha$ by $\kappa$ \\
$a,b$ (—)       & 1.5, 1.0 & Double‑well potential for $\kappa$ \\
$U,V$ (—)       & 1.0, 2.0 & Cross‑coupling gains \\
$W$ (bits s$^{-1}$) & 0.5 & Basal entropy production \\
$\sigma_\xi$ (bits s$^{-1/2}$) & 0.1 & Noise intensity in $E_r$ \\
\bottomrule
\end{tabular}
\end{center}

\subsection*{Code and Data Availability}

All simulation scripts, raw trajectories, plotting notebooks and figure source files are openly available at  
\url{https://github.com/agourakis82/entropic-symbolic-society} (commit 290f158) and archived on Zenodo (10.5281/zenodo.16682785).

% ---------- FIGURES ----------
\begin{figure}[ht]
\centering
\includegraphics[width=\linewidth]{figures/Fig1_collapse.png}
\caption{\textbf{Collapse dynamics.} Time‑series of $\alpha$, $\kappa$ and $E_r$ before and after a step‑increase in entropic drive ($W$) at $t=100$.}
\label{fig:collapse}
\end{figure}

\begin{figure}[ht]
\centering
\includegraphics[width=\linewidth]{figures/Fig2_bifurcation.png}
\caption{\textbf{Bifurcation diagram.} Stationary values of $\alpha$, $\kappa$ and $E_r$ as a function of basal entropic drive $W$. A saddle‑node transition occurs at $W_c\approx1.1$.}
\label{fig:bifurcation}
\end{figure}

% References
\bibliographystyle{unsrt}
\bibliography{references}

\end{document}
