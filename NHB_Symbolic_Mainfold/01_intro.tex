

\section*{Introduction}

Human cognition unfolds across a multidimensional landscape shaped not only by neurons and neurotransmitters, but also by symbols, narratives, and recursive self-reference. Despite increasing interest in the dynamics of consciousness, creativity, and neurodiversity, current cognitive models often remain confined to linear or categorical frameworks, unable to fully capture the recursive, context-dependent nature of mental states.

In recent years, research has begun to explore cognition as a dynamic system governed by principles of complexity and entropy \cite{carhart2014entropic, bassett2017network}. Yet, even these approaches often overlook the symbolic architecture of mind—the way in which individuals anchor meaning, generate representational curvature, and navigate between internal coherence and external adaptation.

Here we introduce a theoretical framework based on symbolic manifolds: a cognitive topology defined by three entropic-symbolic parameters—anchoring coefficient ($\alpha$), symbolic curvature ($\kappa$), and recursive entropy ($E_r$). Together, these variables describe the evolving state of a cognitive system as it traverses a symbolic phase space. The proposed model formalizes the intuition that high-level cognition is neither stable nor chaotic, but dynamically self-modulating across an entropic gradient.

To operationalize this framework, we derive a symbolic analogue of the Schrödinger equation, treating mental trajectories as $\gamma(t) = (\alpha(t), \kappa(t), E_r(t))$ evolving over time. Through simulation, we examine how different parameter configurations yield recognizable cognitive profiles—from neurotypical states to gifted and twice-exceptional (2e) singularities—each occupying distinct regions in the symbolic manifold.

Our goal is not merely to describe exceptional minds, but to propose a topology wherein such minds become computationally representable as structured dynamical attractors. This symbolic manifold offers a new lens for interpreting cognitive variability, bridging creativity, neurodivergence, and mental instability under a unified entropic formalism.