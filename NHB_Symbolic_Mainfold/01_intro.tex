\section{Introduction}

\subsection*{Significance Statement}
A unified, low–dimensional representation of symbolic cognition---the \emph{symbolic manifold}---offers a generative grammar for modelling neurodiversity without resorting to rigid diagnostic categories. By formalising how semantic anchoring, associative divergence and recursive entropy interact, our framework provides a testable bridge between brain–entropy theories, computational psychiatry and lived cognitive diversity, with potential translational value for personalised education and mental‑health interventions.

\bigskip
Understanding cognition as \emph{dynamic\,--\,symbolically mediated\,--\,recursive} remains a central challenge in contemporary cognitive science and psychiatry. Decades of research show that brain activity unfolds as metastable, nonlinear patterns \cite{kelso2012multistability,bassett2017network}, yet canonical cognitive models still rely on static taxonomies (e.g., DSM) or sequence‑oriented architectures (e.g., ACT‑R, SOAR, RNNs) that struggle to capture symbolic recursion and contextual fluidity, especially in neurodivergent or twice‑exceptional (2e) populations \cite{silberman2015neurodiversity}. 

\paragraph{Entropy, complexity, and the missing symbolic layer.}
The Entropic Brain Hypothesis (EBH) links elevated neural entropy to altered states of consciousness \cite{carhart2014entropic}, while the Free Energy Principle (FEP) frames cognition as prediction‑error minimisation \cite{friston2010free}. Both emphasise dynamics, yet neither offers an explicit \emph{symbolic topology}: a space in which semantic coherence, associative leaps and self‑referential loops can be formalised together. Empirical work on narrative coherence \cite{graesser1997coherence,marini2005coherence}, semantic‑network flexibility \cite{kenett2014semantic,kenett2019semantic}, and neural entropy \cite{schartner2017evidence} indicates three partially orthogonal axes of variation---coherence, divergence and unpredictability---but lacks an integrative mathematical substrate.

\paragraph{The symbolic manifold.}
We address this gap by introducing a \textbf{symbolic manifold} $\mathbb{S}$: a topological---though metaphorical---state space parameterised by \emph{anchoring coefficient} $\alpha$ (semantic coherence), \emph{symbolic curvature} $\kappa$ (associative divergence) and \emph{recursive entropy} $E_r$ (self‑referential uncertainty). A cognitive trajectory $\gamma(t) = (\alpha(t),\kappa(t),E_r(t))$ thus formalises how thought evolves under coupled nonlinear forces (Section~\ref{sec:model}). These variables can be mapped onto empirical proxies: $\alpha$ via discourse‑coherence metrics and default‑mode network integration \cite{sporns2013network}; $\kappa$ via semantic‑distance dispersion and creativity indices \cite{beaty2016creative}; $E_r$ via signal entropy in fMRI/EEG \cite{carhart2014entropic,pillow2015entropy}.  

\paragraph{From categories to attractor regimes.}
Rather than labelling minds as discrete “types,” we conceptualise \emph{cognitive profiles} (neurotypical, gifted, 2e, collapse‑prone) as dynamical \emph{attractors} within $\mathbb{S}$. This shift aligns with dynamical‑systems psychiatry \cite{vandeleemput2014} and emerging manifold analyses of neural population activity \cite{langdon2023}. By simulating trajectories under different parameterisations (Section~\ref{sec:results}), we reproduce hallmark regimes: stable high‑$\alpha$ neurotypicality, high‑$\kappa$ creative expansion, oscillatory 2e patterns, and collapse scenarios where declining $\alpha$ precipitates runaway $\kappa$ and $E_r$.

\paragraph{Ethical and societal context.}
Formalising neurodiversity entails socio‑ethical responsibility. Manifold‑based profiles risk reification if naively mapped onto individuals. We therefore embed safeguards: (i) profiles are \emph{metastable regimes}, not fixed labels; (ii) contextual variables (culture, stress, support) modulate trajectories; (iii) any clinical or educational application must preserve agency and avoid pathologisation. This stance resonates with neurodiversity advocacy \cite{silberman2015neurodiversity} and aligns with *Nature Human Behaviour*’s requirement for societal‑impact reflection.

\paragraph{Aim of the present work.}
We provide (i) a formal definition of the symbolic manifold and its governing operator $\mathcal{E}$ (Section~\ref{sec:model}); (ii) simulations revealing canonical dynamical regimes and bifurcations (Section~\ref{sec:results}); and (iii) a discussion on empirical validation pathways, theoretical integration with FEP/EBH, and translational horizons for personalised cognition (Section~\ref{sec:discussion}). Our framework aspires to catalyse an interdisciplinary agenda bridging symbolic AI, neurodynamics, and psychosocial scholarship on cognitive diversity.