\section*{Introduction}

Understanding the dynamic nature of cognition remains a central challenge in cognitive science. Traditional models often rely on static categories or linear representations, which fail to capture the fluid, recursive, and symbolically mediated nature of mental processes—particularly in cases of neurodivergence, exceptional cognitive ability, or disorganization.

Recent research has highlighted the role of entropy and complexity in shaping brain activity and cognitive function \cite{carhart2014entropic, bassett2017network}, suggesting that cognition emerges from nonlinear, metastable dynamics. Yet, these frameworks frequently overlook the influence of symbolic structure, contextual grounding, and self-referential feedback in shaping cognitive states.

Here, we propose a novel theoretical framework based on a symbolic manifold—a topological space parameterized by three cognitive-symbolic variables: anchoring coefficient ($\alpha$), symbolic curvature ($\kappa$), and recursive entropy ($E_r$). These variables jointly describe the evolving state of a cognitive agent as a trajectory $\gamma(t)$ in symbolic space, governed by interactions between symbolic coherence, conceptual divergence, and entropic modulation.

Rather than viewing mental states as discrete or pathologized categories, we conceptualize them as structured positions and flows within this manifold. Cognitive profiles—such as neurotypical, gifted, and twice-exceptional (2e)—emerge as attractor regimes, each exhibiting distinct dynamic signatures.

This model enables the simulation of symbolic trajectories and the mapping of cognitive diversity within a unified, generative structure. By integrating symbolic topology with entropic principles, we aim to provide a rigorous and extensible basis for representing cognitive variability, with potential applications in neuropsychology, psychiatry, and computational modeling.