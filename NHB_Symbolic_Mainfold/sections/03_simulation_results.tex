\section*{3. Simulation Results}

To evaluate the expressive capacity of the symbolic manifold $\mathbb{S}$, we conducted discrete-time simulations of trajectories $\gamma(t) = (\alpha(t), \kappa(t), E_r(t))$ under idealized initial conditions representative of four cognitive-symbolic profiles: Neurotypical (NT), Gifted (G), Twice-exceptional (2e), and Collapse-prone (C). Each trajectory was iterated for $T=10{,}000$ discrete time steps to ensure convergence of dynamic metrics. The system evolves via the entropic modulation operator $\mathcal{E}$ (see Section 2), incorporating deterministic nonlinear interactions and parametric noise.

\paragraph{Profile 1: Neurotypical (NT)} shows stable anchoring ($\alpha \approx 0.9$), low symbolic curvature ($\kappa \approx 0.1$), and minimal entropy ($E_r \approx 0.05$). Its trajectory remains confined to a narrow symbolic basin, suggesting low variance in symbolic state over time (Fig.~\ref{fig:gamma_profiles}). *Quantitatively, the NT profile exhibited the smallest fluctuations in $\kappa$ and $E_r$ (standard deviations $\sigma_{\kappa}\approx0.07$, $\sigma_{E_r}\approx0.015$), consistent with its highly stable cognitive regime.*

\paragraph{Profile 2: Gifted (G)} exhibits moderate anchoring ($\alpha \approx 0.6$), elevated curvature ($\kappa \approx 0.75$), and high recursive entropy ($E_r \approx 0.8$). The trajectory demonstrates symbolic divergence without collapse, reflecting nonlinearity and symbolic flexibility (Fig.~\ref{fig:gamma_profiles}). *This profile shows greater variability than NT ($\sigma_{\kappa}\approx0.10$, $\sigma_{E_r}\approx0.10$), indicating a dynamically richer yet still bounded trajectory.*

\paragraph{Profile 3: Twice-exceptional (2e)} oscillates between basins, with $\alpha(t)$ fluctuating between 0.3 and 0.85, and episodic surges in $E_r(t)$. This profile combines phases of symbolic coherence with entropic disruption, simulating cognitive dissonance or oscillatory stability (Fig.~\ref{fig:gamma_profiles}). *Accordingly, the 2e trajectory spans a broad range of the manifold (e.g., $\alpha$ from $\sim0.2$ to $0.85$), with high variability in curvature and entropy ($\sigma_{\kappa}\approx0.28$, $\sigma_{E_r}\approx0.21$), reflecting its alternating stable and unstable phases.*

\paragraph{Profile 4: Collapse-prone (C)} starts at $\alpha_0 = 0.5$ and undergoes progressive degradation due to compounding $E_r$ and $\kappa$. By $t = 50$, $\alpha \rightarrow 0.1$ and $E_r > 0.9$, indicating a symbolic breakdown. This regime reflects unstable symbolic anchoring and associative overload. *Consistently, the collapse-prone trajectory exhibits an initial moderate variability in $\kappa$ ($\sigma_{\kappa}\approx0.10$ from stochastic fluctuations) but an unbounded increase in $E_r$ (reaching a maximum of $\sim1.4$), culminating in a collapse of $\alpha$. The critical collapse threshold $\alpha<0.2$ is reached at $t_c \approx 46$, marking the time to cognitive breakdown in this profile.*

\paragraph{Quantitative Trajectory Metrics:} To compare profiles rigorously, we computed key quantitative metrics from the simulated trajectories. **(i)** *Geometric curvature:* We defined the instantaneous geometric curvature of the trajectory $\gamma(t)$ in $(\alpha,\kappa,E_r)$ as $\mathcal{C}(t) = \frac{\|\dot{\gamma}(t)\times \ddot{\gamma}(t)\|}{\|\dot{\gamma}(t)\|^3}$, measuring how sharply the path bends through the symbolic state space. The mean trajectory curvature $\langle \mathcal{C}\rangle$ was highest for the oscillatory profiles (2e and G), which trace complex, looping paths, and lowest for the collapse-prone profile, whose path is nearly geodesic (straight) during the descent into collapse. The neurotypical path, being almost static, had negligible curvature on a macro scale (despite large $\mathcal{C}(t)$ spikes in a differential sense due to minimal $\dot{\gamma}$). **(ii)** *Symbolic variance:* We quantified the variability of each coordinate as a proxy for symbolic state dispersion. As noted, the NT profile showed minimal variance in both $\kappa$ and $E_r$ (tightly clustered state), the G profile had moderate variance, and the 2e profile the highest variance across both dimensions. The collapse profile displayed a dichotomy: relatively constrained $\kappa$ variance versus a dramatic increase in $E_r$ variance over time. These geometric and variability measures reinforce the qualitative regime distinctions, indicating that NT and G trajectories reside in narrow attractor basins, whereas 2e explores a much larger region of $\mathbb{S}$ and C follows a one-way path out of any basin.

\paragraph{Entropy Accumulation and Stability:} We next examined entropic load and stability across profiles. The *cumulative entropy* intake, computed as the time-integral $\int_0^T E_r(t)\,dt$, was greatest for the collapse-prone profile (indicative of sustained high entropy over 10k steps) and lowest for the neurotypical profile. Numerically, the collapse trajectory accrued roughly three times the total entropy of the NT trajectory, with Gifted and 2e profiles at intermediate levels. This ordering aligns with cognitive stability: high integrated entropy correlates with eventual breakdown, whereas low entropy supports persistent order. In terms of *stability duration*, only the collapse-prone profile reached the defined collapse criterion $\alpha < 0.2$, doing so by $t_c \approx 46$ (confirming the rapid decline after about 5\% of the simulation). The 2e profile, despite momentary dips of $\alpha$ to $\sim0.2$, always rebounded before sustained collapse, and the G and NT profiles maintained $\alpha(t)\gg0.2$ throughout. Thus, the model exhibits a clear stability hierarchy: NT $>$ G $>$ 2e $>$ C, in line with their anchoring levels and entropic exposures.

\paragraph{Collapse–Recovery Scenario:} We simulated an external dampening event at $t = 70$, abruptly reducing $E_r$ and increasing $\alpha$ by intervention terms $\delta_\alpha$ and $\delta_E$. This perturbation emulates a restorative intervention (e.g., a sudden contextual support or therapeutic input) applied at the brink of collapse. The trajectory returns to a bounded symbolic region, illustrating restoration of coherence under entropic compression (Fig.~\ref{fig:collapse_recovery}). *Notably, $\alpha$ is elevated from $\approx0.1$ to $\approx0.4$ immediately after the intervention, partially re-anchoring the system, while $E_r$ is concomitantly lowered, alleviating the entropic overload. This results in the post-intervention trajectory resembling a stabilized regime akin to a neurotypical or mildly gifted state, rather than continuing toward chaos.* The collapse–recovery simulation highlights the model’s capacity to incorporate resilience and external modulation, reinforcing the idea that even a collapse-prone cognitive trajectory can be steered back into a stable manifold region through targeted reductions in entropy and boosts to anchoring.

\paragraph{Symbolic Transition Heatmap:} To visualize the state-space dynamics, we generated heatmaps of symbolic state occupancy and transitions. In the $(\alpha, E_r)$ plane (with $\kappa$ in a moderate range), the distribution of trajectory points (Fig.~\ref{fig:symbolic_heatmap}) revealed two dominant basins corresponding to high-anchoring/low-entropy states versus low-anchoring/high-entropy states. Neurotypical and Gifted profiles concentrated almost entirely in the stable high-$\alpha$, low-$E_r$ region (upper-left corner of the heatmap), whereas the Collapse trajectory occupies the opposite extreme as it approaches $\alpha\to0$ with $E_r$ high (lower-right region). The 2e profile shows a broad spread across the map, frequently traversing between the two basins—its points form a bridged band connecting the stable and unstable extremes. The relative sparsity of points in the mid $\alpha$, mid $E_r$ zone suggests a transitional saddle region: trajectories either remain well-anchored or tend toward collapse, with fewer sustained states in between. This symbolic transition map underscores an inverse relationship between anchoring and entropy in the model: maintaining high $\alpha$ inherently constrains $E_r$, whereas excessive entropy erodes $\alpha$, driving the system toward the collapse basin.

\paragraph{Bifurcation Analysis:} We further probed the model’s nonlinear dynamics via a simulated bifurcation scenario, focusing on how increasing symbolic curvature $\kappa$ can destabilize anchoring $\alpha$. In this experiment, $\kappa(t)$ was slowly ramped up as an exogenous driver while holding other inputs nominal (mimicking intensifying cognitive complexity), and we observed the resulting behavior of $\alpha$. The bifurcation diagram (Fig.~\ref{fig:kappa_bifurcation}) illustrates that up to a critical curvature threshold (around $\kappa \sim 1.0$ in the example trajectory), the anchoring $\alpha$ remains relatively steady. However, as $\kappa$ crosses this threshold (marked by $t \approx 60$ in the simulation), the system can no longer maintain the previous $\alpha$ level: a sharp drop in $\alpha$ ensues, indicating a loss of stability. This curvature-induced collapse is analogous to a saddle-node bifurcation: beyond a certain point, no stable high-$\alpha$ equilibrium exists, forcing $\alpha$ to a lower branch (the collapsed state). *In the plotted simulation, $\alpha$ was artificially held constant pre-threshold for clarity, but in a fully coupled run we expect $\alpha$ to plummet once $\kappa$ exceeds the bifurcation point.* The analysis suggests that heightened symbolic curvature (e.g. increasingly nonlinear, tangential cognitive leaps) can push an otherwise anchored cognitive state into a divergent regime, providing a theoretical link between cognitive complexity and breakdown.

\paragraph{Symbolic Topology and Clustering:} Visualizing the full 3D trajectory $\gamma(t)$ in the manifold $(\alpha, \kappa, E_r)$ provides an integrated picture of each profile’s dynamic regime. Each cognitive profile traces a distinct path through this symbolic state space, revealing organized transitions and natural clustering of states (Fig.~\ref{fig:trajectory_3D}). For example, the NT trajectory stays tightly clustered around a single attractor-like region (high $\alpha\approx0.9$, low $\kappa,\!E_r$), reflecting its consistent, well-regulated cognitive state. The Gifted trajectory forms an extended loop at moderate $\alpha$ and high $\kappa,E_r$—an expansive but still bounded orbit indicative of sustained exploratory dynamics without collapse. In contrast, the 2e trajectory covers two qualitatively different clusters: one cluster at higher $\alpha$ (with lower $E_r$) corresponding to episodes of coherence, and another at lower $\alpha$ (with heightened $E_r$) corresponding to disorganized phases. Unsupervised clustering of the 2e trajectory points (e.g., using a density-based algorithm) indeed separates these two groups, with a transitional pathway between them, corroborating the notion of bistability or mode-switching in the 2e profile. The collapse-prone trajectory initially meanders in a mid-range cluster (intermediate $\alpha\sim0.5$), then veers off toward the far end of the manifold, terminating in a distinct collapse cluster characterized by $\alpha \to 0$ and maximal $E_r$. These clustered structures lend support to the idea of **symbolic attractors** in the cognitive manifold: despite continuous dynamics, the system’s states tend to concentrate in particular regions (attractors or metastable basins), separated by rapid transitions. This topological perspective implies that individual cognitive profiles could be distinguished by their unique configuration of symbolic attractors and transition patterns—effectively a **symbolic cognitive fingerprint** for each profile.

\paragraph{Comparative Positioning:} Table~\ref{tab:model_comparison} situates our framework relative to other cognitive theories. While the Free Energy Principle (FEP) and the Entropic Brain hypothesis emphasize entropy-centric formulations of brain dynamics, the symbolic manifold approach uniquely enables *symbolic* interpretability and explicit modeling of neurodiversity. In contrast to FEP’s global free-energy minimization and Carhart-Harris’s entropic brain framework \cite{carhart2014entropic} linking elevated neural entropy to unconstrained cognition, our model provides a structured state space where entropy ($E_r$) interacts with symbolic order ($\alpha$) and cognitive complexity ($\kappa$). This explicit topology allows us to capture profiles like gifted and 2e—which involve high complexity without pathological collapse—beyond the scope of purely entropy-based accounts.

Together, the above simulations support the hypothesis that symbolic cognitive profiles manifest as structured dynamic regimes in $\mathbb{S}$, offering a generative and interpretable approach to modeling variation in mental states.

\paragraph{Model Robustness and Generalization:} An important question is how sensitive these results are to parameter variations and noise. We found that the qualitative regime distinctions (stable vs. oscillatory vs. collapsing trajectories) are robust across a range of initial conditions and model parameter settings. For instance, moderate changes to the initial anchoring $\alpha_0$ or entropy load did not abolish the existence of the four characteristic profiles—each profile remained topologically identifiable, though specific metric values (e.g. time to collapse or amplitude of oscillations) shifted slightly. The model’s dynamics thus generalize across individuals or conditions: a Neurotypical-like regime consistently emerges when $\alpha$ is sufficiently high and $E_r$ low, and a Collapse-like regime arises when $\alpha$ is low or erodes under high $E_r$ pressure. The transitions between regimes (as seen in the 2e profile) persisted under added noise, indicating that the oscillatory switching is an inherent system property rather than a fine-tuned artifact. That said, our analysis also revealed **sensitivity thresholds**: if key parameters push beyond critical values (e.g., excessive $\eta_r$ driving entropy or insufficient $\eta_\alpha$ sustaining anchoring), even a nominally stable profile can tip into collapse. Likewise, reducing noise damping can exaggerate oscillatory behavior in the 2e profile or induce sporadic mini-collapses. These findings underscore that while the symbolic manifold framework is structurally stable, the precise boundaries between cognitive regimes depend on parameter settings—mirroring how individual differences or environmental stressors might shift a person’s cognitive trajectory closer to or further from a collapse threshold.

\paragraph{Translational Implications:} By bridging symbolic dynamics with cognitive phenotypes, our model offers several translational insights and testable hypotheses. The collapse-prone simulation, for example, may be viewed as a stylized model of acute cognitive decompensation (such as a psychotic break or a high-stress cognitive collapse), with the $\alpha$ variable capturing loss of structured thought and $E_r$ representing runaway neural entropy or noise in neural signaling [oai_citation:0‡file-vewdoov2h7pp67c8u4svrl](file://file-Vewdoov2H7PP67C8u4svrL#:~:text=Recent%20research%20has%20highlighted%20the,feedback%20in%20shaping%20cognitive%20states) [oai_citation:1‡file-vewdoov2h7pp67c8u4svrl](file://file-Vewdoov2H7PP67C8u4svrL#:~:text=Rather%20than%20viewing%20mental%20states,each%20exhibiting%20distinct%20dynamic%20signatures). The successful *recovery intervention* (Fig.~\ref{fig:collapse_recovery}) conceptually aligns with therapeutic actions—such as grounding techniques or medication—that reduce cognitive entropy (chaotic thought patterns) and re-establish a coherent narrative or focus (increase in $\alpha$). In the oscillatory 2e profile, the alternating high- and low-anchoring phases resonate with observations of twice-exceptional individuals who cycle between periods of high cognitive performance and disruptive lapses (analogous, perhaps, to alternating attention/hyperfocus and dysregulation). This model suggests that such individuals occupy a fringe of stability in the cognitive manifold, where slight perturbations can induce shifts between functional and dysfunctional states. Notably, the abstract variables $\alpha, \kappa, E_r$ correspond to potential empirical measures: **anchoring** $\alpha$ could be related to semantic coherence or focus (e.g. coherence in speech or goal-directed behavior [oai_citation:2‡file-esh1mryzesc7r3hq1u5ni5](file://file-EsH1mRyZEsC7R3HQ1u5Ni5#:~:text=%5Cbegin,itemize)), **symbolic curvature** $\kappa$ might map onto the propensity for nonlinear thought leaps or divergent thinking (e.g. remote association generation), and **recursive entropy** $E_r$ connects to the unpredictability of cognitive or neural activity (e.g. entropy of EEG signals or fMRI BOLD fluctuations [oai_citation:3‡file-vewdoov2h7pp67c8u4svrl](file://file-Vewdoov2H7PP67C8u4svrL#:~:text=Recent%20research%20has%20highlighted%20the,feedback%20in%20shaping%20cognitive%20states)). Thus, the model provides a framework for interpreting individual differences in terms of measurable cognitive dynamics. In a clinical research context, one could imagine identifying a person’s location in the symbolic manifold via cognitive testing or neuroimaging (for instance, assessing their “anchoring” through attention stability and their neural entropy through EEG complexity). Such a mapping could help predict susceptibility to cognitive collapse or oscillatory instability, offering a translational pathway to inform personalized interventions. Ultimately, by capturing the interplay between order (anchoring), complexity (curvature), and disorder (entropy), the symbolic manifold model has the potential to unify diverse cognitive phenomena—ranging from the structured focus of neurotypical thought to the edge-of-chaos explorations of gifted minds and the brink-of-collapse states in certain psychiatric conditions—within a single, interpretable topological framework.
