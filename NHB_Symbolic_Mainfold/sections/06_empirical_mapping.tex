\section*{Empirical Mapping}
\noindent\textbf{Pilot Experimental Protocol:} To validate the symbolic manifold model, we outline a pilot study combining electrophysiology, neuroimaging, and behavior. Participants from diverse cognitive profiles (e.g. neurotypical, gifted, 2e) would undergo **EEG** recordings, **fMRI** scans, and naturalistic **free speech** tasks. EEG data (including resting-state and task-evoked activity) provide high-temporal-resolution brain dynamics; fMRI measures distributed brain connectivity; and speech samples enable quantitative discourse analysis. By synchronizing these modalities, we aim to capture the multi-scale signatures of a participant’s trajectory on the symbolic manifold in real time.

\noindent\textbf{Mapping Symbolic Variables to Data:} We propose specific empirical proxies for each model dimension $(\alpha, \kappa, E_r)$, as summarized in Table~1. The **anchoring coefficient** $\alpha$ (symbolic stability/coherence) can be indexed by measures of semantic or narrative coherence in speech and thought. For example, coherence metrics from free speech (tracking how on-topic and logically connected a person's narrative is) serve as one proxy [oai_citation:4‡file-3pov5cckyplhcce8hsdken](file://file-3pov5cCkypLHccE8hSDkEN#:~:text=%40article,1). Empirically, reduced semantic coherence in speech has been linked to thought disorganization and impending psychosis [oai_citation:5‡file-3pov5cckyplhcce8hsdken](file://file-3pov5cCkypLHccE8hSDkEN#:~:text=%40article,1), aligning with $\alpha$ low values. Neurally, $\alpha$ may also correlate with stable default-mode network (DMN) engagement and hippocampal co-activation during memory retrieval, reflecting well-anchored thought. The **symbolic curvature** $\kappa$ (divergence or nonlinear associative thinking) can be mapped to network-level metrics of idea generation and connectivity. One proxy is the degree of semantic network spread or semantic distance between concepts generated by an individual [oai_citation:6‡file-3pov5cckyplhcce8hsdken](file://file-3pov5cCkypLHccE8hSDkEN#:~:text=%40article,2014). Individuals high in creativity (analogous to higher $\kappa$) exhibit more diffuse, non-linear associations in semantic memory tests [oai_citation:7‡file-3pov5cckyplhcce8hsdken](file://file-3pov5cCkypLHccE8hSDkEN#:~:text=%40article,2014). Neuroimaging proxies for $\kappa$ could include measures of brain network segregation vs. integration (e.g. global efficiency or hub connectivity in associative networks) and perhaps oscillatory markers of divergent thought (such as transient alpha-band power changes in the posterior cingulate cortex during mind-wandering). Finally, **recursive entropy** $E_r$ (degree of symbolic uncertainty/novelty) can be tied to physiological entropy and complexity measures. We anticipate that high $E_r$ states correspond to elevated neural signal diversity and unpredictability. In EEG/MEG data, this can be quantified by algorithms like Lempel–Ziv complexity or entropy rate of brain signals [oai_citation:8‡file-3pov5cckyplhcce8hsdken](file://file-3pov5cCkypLHccE8hSDkEN#:~:text=title%20%3D%20,10.1038%2Fsrep46421). Notably, psychedelic brain states—considered “entropic” brain states—show increased spontaneous neural signal diversity [oai_citation:9‡file-3pov5cckyplhcce8hsdken](file://file-3pov5cCkypLHccE8hSDkEN#:~:text=title%20%3D%20,10.1038%2Fsrep46421), mirroring an increase in $E_r$. Additionally, perturbational complexity index (PCI) from TMS-EEG and other entropy-based indices of brain activity serve as candidate biomarkers for $E_r$. **Table~1** lists these putative mappings, which ground our theoretical variables in observable phenomena across modalities.

\noindent\textbf{Analysis and Topotype Clustering:} The pilot data will be analyzed to test whether the hypothesized **topotypes** (attractor profiles) manifest as distinct clusters in the empirical measurements. First, each participant (or each cognitive state episode) will be represented as a point in a three-dimensional feature space defined by the chosen proxies for $\alpha$, $\kappa$, and $E_r$. We will apply statistical analyses to examine group differences and correlations: for instance, we expect neurotypical individuals to show significantly higher coherence (higher $\alpha$ proxy) compared to a collapse-prone group, and gifted/creative individuals to exhibit higher divergence ($\kappa$ proxy) and entropy ($E_r$ proxy) relative to controls. Beyond group comparisons, we will use unsupervised learning (e.g., $k$-means or hierarchical clustering) to see if the data naturally cluster into the model’s predicted regimes. We anticipate clusters corresponding to **high-$\alpha$/low-$E_r$** (neurotypical-like), **high-$\kappa$/high-$E_r$** (gifted-like), **oscillatory $\alpha$ with spikes in $E_r$** (2e, alternating between modes), and **low-$\alpha$/high-$E_r$** (collapse-prone). Such clustering would empirically delineate the **topotypes** as regions in the data-driven space, analogous to attractor basins in the symbolic manifold. Finally, we plan to employ time-series analyses to capture dynamics: e.g., monitoring if an individual’s $\alpha$ proxy drops precipitously before a “collapse” event (early-warning signals of destabilization), or if temporary boosts in $\kappa$ and $E_r$ accompany creative breakthroughs. Statistical significance for differences will be assessed (with appropriate corrections), and cluster robustness will be validated via techniques like silhouette analysis and cross-validation. This **Empirical Mapping** approach provides a roadmap to link our theoretical model to measurable brain-and-behavior patterns, setting the stage for rigorous validation of the symbolic manifold framework.

% Box: Glossary of key terms in the Symbolic Manifold framework
\begin{table}[t]
\caption{\textbf{Glossary of Key Terms.} Brief definitions of the principal terms and variables introduced in the symbolic manifold model, for interdisciplinary readers.}
\label{tab:glossary}
\renewcommand{\arraystretch}{1.2}
\begin{tabular}{p{3.2cm} p{12cm}}
\textbf{Term} & \textbf{Definition} \\ \hline
$\alpha$ (anchoring coefficient) & A scalar variable representing semantic coherence or groundedness of thought. High $\alpha$ indicates well-anchored, contextually coherent cognition (focused, organized thought), whereas low $\alpha$ denotes weak anchoring (disorganized or fragmented thought). \\
$\kappa$ (symbolic curvature) & A measure of divergent associative thinking or nonlinear semantic “curvature.” High $\kappa$ corresponds to broad, imaginative or tangential leaps in thought (high cognitive flexibility), while low $\kappa$ implies linear, conventional associations (low divergence). \\
$E_r$ (recursive entropy) & The degree of uncertainty, novelty, or randomness in the self-referential symbolic process. Higher $E_r$ signifies a more unpredictable and complex flow of thoughts (rich recursive entropy, akin to chaotic or entropic cognition), whereas lower $E_r$ implies more predictable, stable symbolic patterns. \\
\textit{Symbolic manifold} & The three-dimensional abstract state-space defined by coordinates $(\alpha,\kappa,E_r)$. Each point in this manifold represents a cognitive state, and trajectories through the manifold describe the evolution of cognition over time. It provides a topological landscape for mental activity, with distinct regions corresponding to different cognitive regimes. \\
\textit{Topotype} & A dynamic attractor profile or “mode” of cognition characterized by a stable combination of $(\alpha,\kappa,E_r)$ values. Examples in this work include the **neurotypical** topotype (high $\alpha$, low $E_r$), the **gifted** topotype (elevated $\kappa$ and $E_r$), the **2e (twice-exceptional)** topotype (oscillation between high and low $\alpha$ states), and the **collapse-prone** topotype (low $\alpha$, high $\kappa$, high $E_r$ leading to instability). Topotypes are not static categories but metastable dynamic regimes within the symbolic manifold. \\ \hline
\end{tabular}
\end{table}