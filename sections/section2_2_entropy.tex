\section{Entropy as a Vector of Organisation}

Contrary to the simplistic view that reduces it to chaos or the dissolution of order, entropy can be understood as a deep engine of systemic reorganisation. In its most precise definition, it measures the number of possible configurations of a system — that is, its microscopic degrees of freedom. The greater the entropy, the broader the space of possibilities. And it is precisely within this expanded space that the new can emerge.

According to the second law of thermodynamics, entropy tends to increase in any isolated system. Everything in the universe spontaneously flows towards states of greater apparent disorder — or, more accurately, towards statistically more probable configurations. This entropic movement does not imply the absence of structure, but rather a continuous transition between potential arrangements. In open systems — such as living organisms, minds, or societies — entropic pressure does not destroy, but compels reorganisation.

Entropy, therefore, does not impede order — it demands a higher order.

The emergence of life, for instance, can be interpreted as an adaptive response to incoming energy flows — a more efficient means of dissipating energy. This is what Jeremy England calls “dissipative adaptation”: structures that persist are those capable of absorbing and dispersing energy more effectively, reorganising themselves under entropic pressure.

Here, the parallel with fractals becomes inevitable. Fractals are not frozen structures — they are dynamic, iterative patterns that reorganise at multiple scales while preserving structural identity. Likewise, entropy does not lead to pure fragmentation, but forces the system to regroup — now at a new level of complexity.

The human body itself exemplifies this principle: the branching architecture of the lungs, the vascular system, neural networks — all obey fractal patterns that maximise energetic efficiency while maintaining structural coherence under increasing entropy.

Society, in turn, can be seen as a meta-layer of these dynamics. It is neither static nor linear, but an iterative expression shaped by the same thermodynamic forces that govern the cosmos. Thus, entropy is not the antithesis of order — it is its invisible womb.