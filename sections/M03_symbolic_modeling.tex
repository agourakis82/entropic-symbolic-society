\subsection*{3.1 Fractal States and the Entropic Continuum of Mental Singularities}

The classical taxonomies of psychiatry, especially those codified in the DSM-5, treat mental disorders as discrete deviations from a neurotypical baseline. However, such models neglect the nonlinear, recursive, and scale-free dynamics underpinning cognitive phenomena (Rapp and Babloyantz, 1985; Werner, 2007; Goldberger et al., 2002). We propose a symbolic-entropic reframing, where states of mind—including those labeled as pathological—are conceptualized as attractors within a fractal manifold whose topology is modulated by sociocultural symbolic fields (Hernández et al., 2023; Borsboom, 2017).

Each mental configuration occupies a point of entropic curvature—zones of symbolic recursion that can amplify, stabilize, or disintegrate. Psychosis may represent hyper-entropic bifurcations, a breakdown in symbolic coherence; depression may reflect low-entropy attractors with semantic rigidity; while giftedness may manifest as a symbolic singularity capable of high entropic coherence (Carhart-Harris et al., 2014; Tagliazucchi et al., 2016; Silverman, 2009).

Critically, these configurations are not merely intrapsychic: their stability and classification are shaped by cultural topology, narrative saturation, and symbolic affordance (Deacon, 2011; Lotman, 1990). Features traditionally pathologized—such as hypersensitivity, symbolic overproduction, or recursive abstraction—may indicate high-functioning neurodivergence (Rinn and Reynolds, 2012; Neihart et al., 2002).

In this light, pathology is not defined by deviation per se, but by the collapse of symbolic recursion under entropic tension. Singularities become zones of cognitive acceleration that, depending on anchoring conditions, may either crystallize into insight or dissolve into dysfunction. The symbolic-mental continuum thus requires a topology of phase transitions—not binary categories—governed by recursive coherence, cultural curvature, and semantic resonance (Friston, 2010; Bateson, 1979; Prigogine, 1980).

\subsection*{3.2 Symbolic Schrödinger Dynamics of Mental States}

Building on the entropic manifold above, we now introduce a symbolic generalization of the Schrödinger equation to model the evolution of mental states under symbolic and sociocultural modulation.

Let \( \Psi_s(x, t) \) represent the symbolic wavefunction of a cognitive agent, encoding recursive semantic potential at symbolic position \( x \) and time \( t \). Each \( x \) maps onto configurations in the agent's internal semantic space. The amplitude of \( \Psi_s \) reflects the coherence, recursion depth, and symbolic density of mental activity.

The potential field \( V_s(x, t) \) denotes the sociocultural-symbolic landscape—composed of expectations, norms, trauma imprints, epistemic constraints, and symbolic supports—modulating the curvature of cognition. This symbolic potential is entangled with cultural semiotics, as theorized by Lotman (1990) and Deacon (2011).

Analogous to Planck’s constant, \( \hbar \) encodes the minimal unit of recursive interpretation—symbolic granularity—while \( m \) reflects the inertia of symbolic structures (e.g., cognitive rigidity or belief anchoring).

We thus propose:

\[
i\hbar \frac{\partial \Psi_s(x, t)}{\partial t} = \left( -\frac{\hbar^2}{2m} \nabla^2 + V_s(x, t) \right) \Psi_s(x, t)
\]

Here, \( \nabla^2 \) models the symbolic diffusion operator: the spread of meaning across conceptual proximity. This aligns with cognitive diffusion models such as those proposed by Busemeyer and Bruza (2012) and with the free-energy dynamics articulated by Friston (2010).

In depressive states, deep negative symbolic potentials constrain \( \Psi_s \), localizing the mind in semantic minima. Conversely, psychotic disintegration may correspond to decoherence triggered by symbolic hyper-excitation (Carhart-Harris et al., 2014; Tagliazucchi et al., 2016).

Giftedness, by contrast, sustains coherence across high entropic gradients, exhibiting stable symbolic standing waves anchored in recursive networks (Silverman, 2009; Rinn and Reynolds, 2012). These high-dimensional attractors hold complexity without collapse.

This formalism enables the mapping of phase transitions across symbolic-mental topologies—linking thermodynamic entropy, semantic curvature, and recursive cognition within a symbolic physics of the mind.