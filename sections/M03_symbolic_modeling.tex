\subsection*{3.4 Diagnostic Topology and the Entropic Mapping of Mental States}

Having developed a symbolic-physical framework to describe mental states, we now explore its translational potential: can subjective experience be mapped within a structured topological space that respects both entropic recursion and sociocultural curvature? This section proposes a diagnostic topology wherein psychiatric states are not categorized discretely but positioned within a dynamically deformable manifold of symbolic entropy, cultural constraint, and recursive coherence.

Let us consider the Diagnostic and Statistical Manual of Mental Disorders (DSM-5) not as a definitive classification system but as a projection of a higher-dimensional entropic field onto a low-dimensional clinical map. Much like how a Mercator projection distorts the geometry of a sphere, psychiatric taxonomies flatten the non-Euclidean, recursively entangled symbolic field of mind into a categorical grid. Consequently, conditions such as autism spectrum disorder, major depression, bipolarity, schizophrenia, ADHD, and giftedness may represent proximal zones in entropic-space, though rendered disjoint by diagnostic flattening.

In this symbolic topology, each mental configuration is defined by three primary coordinates: symbolic entropy (\( S_s \)), recursive anchoring (\( R_s \)), and cultural modulation (\( C_s \)). The resulting point \( M_s = (S_s, R_s, C_s) \) situates a subjectivity in phase-space. Diagnostic states become basins of symbolic stability or instability. High \( S_s \) with low \( R_s \) and negative \( C_s \) may correspond to psychosis; low \( S_s \) and \( R_s \) under restrictive \( C_s \) may align with depression; high \( S_s \), \( R_s \), and variable \( C_s \) may model gifted states in flux.

In this model, the trajectory of a person through psychological space is path-dependent and nonlinear. Clinical states may emerge not as fixed entities but as topological distortions of an evolving symbolic field. Friston's free-energy principle supports this: states of mental suffering may reflect energetic divergences between predictive priors and incoming symbolic perturbations \cite{friston2010}. Similarly, Borsboom's network models propose that symptoms stabilize one another through recursive reinforcement \cite{borsboom2017}, suggesting that diagnostic labels are local gravitational basins, not intrinsic singularities.

This reframing invites a new diagnostic epistemology—one where transitions, bifurcations, and attractor states can be visualized and possibly computed. Future work may link topological data analysis (TDA), graph signal processing, and semantic field theory to construct such a diagnostic cartography. Crucially, it opens space for integrating giftedness and divergence as topological configurations, not merely as outliers to be normalized but as structural singularities to be understood.

Rather than impose grids onto minds, we propose to map the manifold itself.

The classical taxonomies of psychiatry, especially those codified in the DSM-5, treat mental disorders as discrete deviations from a neurotypical baseline. However, such models neglect the nonlinear, recursive, and scale-free dynamics underpinning cognitive phenomena \cite{rapp1985, werner2007, goldberger2002}. We propose a symbolic-entropic reframing, where states of mind—including those labeled as pathological—are conceptualized as attractors within a fractal manifold whose topology is modulated by sociocultural symbolic fields \cite{hernandez2023, borsboom2017}.

Each mental configuration occupies a point of entropic curvature—zones of symbolic recursion that can amplify, stabilize, or disintegrate. Psychosis may represent hyper-entropic bifurcations, a breakdown in symbolic coherence; depression may reflect low-entropy attractors with semantic rigidity; while giftedness may manifest as a symbolic singularity capable of high entropic coherence \cite{carhart2014, tagliazucchi2016, silverman2009}.

Critically, these configurations are not merely intrapsychic: their stability and classification are shaped by cultural topology, narrative saturation, and symbolic affordance \cite{deacon2011, lotman1990}. Features traditionally pathologized—such as hypersensitivity, symbolic overproduction, or recursive abstraction—may indicate high-functioning neurodivergence \cite{rinn2012, neihart2002}.

In this light, pathology is not defined by deviation per se, but by the collapse of symbolic recursion under entropic tension. Singularities become zones of cognitive acceleration that, depending on anchoring conditions, may either crystallize into insight or dissolve into dysfunction. The symbolic-mental continuum thus requires a topology of phase transitions—not binary categories—governed by recursive coherence, cultural curvature, and semantic resonance \cite{friston2010, bateson1979, prigogine1980}.

In what follows, we transition from conceptual modeling to practical epistemology. M04 will articulate how this diagnostic topology interfaces with existing psychiatric paradigms, and how it may offer a computational framework for early detection, symbolic divergence tracking, and topological biomarkers of giftedness and collapse.

\subsection*{3.2 Symbolic Schrödinger Dynamics of Mental States}

Building on the entropic manifold above, we now introduce a symbolic generalization of the Schrödinger equation to model the evolution of mental states under symbolic and sociocultural modulation.

Let \( \Psi_s(x, t) \) represent the symbolic wavefunction of a cognitive agent, encoding recursive semantic potential at symbolic position \( x \) and time \( t \). Each \( x \) maps onto configurations in the agent's internal semantic space, which itself is embedded within a higher-order symbolic manifold. The amplitude of \( \Psi_s \) reflects the coherence, recursion depth, and symbolic density of mental activity — a construct that parallels quantum amplitude but is not meant to be interpreted probabilistically in this context, but structurally and semantically.

The potential field \( V_s(x, t) \) denotes the sociocultural-symbolic landscape—composed of expectations, norms, trauma imprints, epistemic constraints, and symbolic supports—modulating the curvature of cognition. This symbolic potential is entangled with cultural semiotics and narrative structures, shaping how cognitive trajectories unfold or collapse \cite{lotman1990, deacon2011}.

Analogous to Planck’s constant, \( \hbar \) encodes the minimal unit of recursive interpretation—symbolic granularity—while \( m \) reflects the inertia or rigidity of symbolic structures (e.g., dogmas, affective fixations, or tightly-bound identity constructs).

We thus propose:

\[
i\hbar \frac{\partial \Psi_s(x, t)}{\partial t} = \left( -\frac{\hbar^2}{2m} \nabla^2 + V_s(x, t) \right) \Psi_s(x, t)
\]

Here, \( \nabla^2 \) models the symbolic diffusion operator: the spread or dissipation of meaning across conceptual proximity. This formalism parallels cognitive diffusion models such as those proposed by \cite{busemeyer2012} and the free-energy minimization dynamics articulated by \cite{friston2010}. In our symbolic Schrödinger analogy, diffusion does not represent thermodynamic heat but rather the semantic drift of interpretive content — i.e., how symbols and narratives generalize or lose coherence when their recursive loops are not anchored.

Importantly, this formulation is not intended to reduce consciousness to a quantum mechanical system per se, but to provide a formal language that captures the dynamics of symbolic tension, entropic pressure, and recursive stability in the cognitive domain. While this formulation employs analogical language, it is consistent with the broader tradition of mathematical formalism in complex systems theory, where structural mappings illuminate dynamics beyond direct physical isomorphism. The present framework is best understood as a *mathematical allegory* for recursive symbolic dynamics within entropically modulated sociocognitive spaces.

In depressive states, deep negative symbolic potentials constrain \( \Psi_s \), localizing the wavefunction in semantic minima that resist recursive propagation. These regions correspond to affective flatness, cognitive rigidity, and diminished symbolic production. Conversely, psychotic disintegration may correspond to decoherence triggered by symbolic hyper-excitation — overactivation of recursive loops without structural containment, as seen in altered states of consciousness and schizophrenic breaks \cite{carhart2014, tagliazucchi2016}.

Giftedness, by contrast, sustains coherence across high entropic gradients, exhibiting stable symbolic standing waves anchored in multi-domain recursive networks \cite{silverman2009, rinn2012}. These high-dimensional attractors hold symbolic complexity without collapse — a trait often misunderstood or pathologized due to its divergence from neurotypical anchoring norms.

By formalizing symbolic recursion in a Schrödinger-type structure, we allow the space of mental states to be mapped not through linear taxonomies, but through dynamic symbolic fields capable of bifurcation, coherence, and collapse. This opens the possibility of future computational simulations or empirical mappings of symbolic phase states, potentially linking linguistic data, affective measures, and cognitive entropy to symbolic field topologies — a path aligned with emerging paradigms in computational psychiatry and neurophenomenology.
